\documentclass[9pt,a4paper]{report}

\usepackage[T1]{fontenc}
\usepackage[polish]{babel}
\usepackage[utf8]{inputenc}
\usepackage{graphicx}
\usepackage[export]{adjustbox}

\graphicspath{ {./media/} }

\usepackage[margin=1.2in]{geometry}

\usepackage{hyperref}
\hypersetup{
    colorlinks=true,
    linkcolor=black,
    filecolor=magenta,      
    urlcolor=blue,
}

\usepackage{amsfonts}
\usepackage{titlesec}

\titleformat{\chapter}[hang] 
{\normalfont\huge\bfseries}{\chaptertitlename\ \thechapter:}{1em}{} 
\titleformat*{\section}{\Large}
\begin{document}

\title{\Huge Projekt pracy i rozwoju sekcji technicznej samorządu uczniowskiego w latach 2020/2021.}
\author{Maciej Tracz \\\\Technikum Mechatroniczne nr 1 w Warszawie}
\date{Rok 2020}
\maketitle

\newpage
\tableofcontents
\newpage

\chapter{Struktura}

\section{Usunięcie starego systemu}
Budowa sekcji stanowi najważniejszą rolę w jej funkcjonowaniu. Wpływa na pracę każdego członka i z niej wynikają wszystkie późniejsze decyzje organizacyjne. Z tego względu na nowo warto przemyśleć to jak aktualnie wygląda Nasza sekcja i jakie niesie to za sobą konsekwencje.\\\\
Sekcja pomimo swojego potencjału oraz uwagi zdobytej na początku roku nie odniosła sukcesu. Osoby wstępnie zapisanie do poszczególnych podsekcji nie wykazywały inicjatywy o co nie można ich dziwić, gdyż to po Naszej stronie stoi zapewnić odpowiednie warunki dla takich inicjatyw.\\Członkowie, którzy dołączyli do projektów trwających do teraz, pozostają w stosunkowej izolacji, zajmując się swoimi sprawami bez poszerzania swoich zagadnień. Zmniejsza to też szanse innych na niesienie pomocy.\\\\
Te właśnie problemy wynikają w znacznej mierze z budowy sekcji oraz braku inicjatywy samorządu na jej rozbudowanie.
\section{Wprowadzenie podziału na Organizację i Współpracę}
\section{Utworzenie małego sektora Organizacji projektów technicznych}
\section{Utworzenie systemu Współpracy uczniów w projektach.}
\section{Podział Współpracy na różne wymiary grup.}

\chapter{Praca (Organizacja)}

\section{Wyznacznie struktury zarządzania projektami}
\section{Dobór osób zarzadzających pracą}
\section{Dostosowanie obowiązków i wymagań stanowisk do struktury}
\section{Wyznaczenie osób opiekujących się istniejącymi technologiami}

\chapter{Praca (Współpraca)}

\section{Wyznacznie metod współpracy uczniów}
\section{Organizacja spotkań sekcji}
\section{Prezentacja dostępnych projektów}
\section{Prezentacja możliwych warsztatów lub prelekcji}
\section{Oferty współpracy nad subprojektami/własnymi projektami}

\chapter{Dostęp i informacja}

\section{Wyznaczenie metod prezentowania zasobów sekcji}
\section{Stworzenie kampanii promującej udział w spotkaniach}
\section{Prezentacja dostępnych projektów, grafiku spotkań oraz warsztatów/prelekcji}
\section{Stworzenie formularza uczestnictwa w projektach lub nauczaniu}
\section{Umożliwienie proponowania projektów/wykorzystania technologii}

\chapter{Bonus}

\section{Zadania tygodniowe na dodatkowe oceny}
\section{Projekty wewnątrz-samorządowe}

\end{document}