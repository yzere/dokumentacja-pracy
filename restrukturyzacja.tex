\documentclass[9pt,a4paper]{report}

\usepackage[T1]{fontenc}
\usepackage[polish]{babel}
\usepackage[utf8]{inputenc}
\usepackage{graphicx}
\usepackage[export]{adjustbox}

\graphicspath{ {./media/} }

\usepackage[margin=1.2in]{geometry}

\usepackage{hyperref}
\hypersetup{
    colorlinks=true,
    linkcolor=black,
    filecolor=magenta,      
    urlcolor=blue,
}

\usepackage{amsfonts}
\usepackage{titlesec}

\titleformat{\chapter}[hang] 
{\normalfont\huge\bfseries}{\chaptertitlename\ \thechapter:}{1em}{} 
\titleformat*{\section}{\Large}
\begin{document}

\title{\Huge Projekt pracy i rozwoju sekcji technicznej samorządu uczniowskiego w latach 2020/2021.}
\author{Maciej Tracz \\\\Technikum Mechatroniczne nr 1 w Warszawie}
\date{Rok 2020}
\maketitle

\newpage
\tableofcontents
\newpage

\chapter{Struktura}

\section{Zmiana systemu}
Budowa sekcji stanowi najważniejszą rolę w jej funkcjonowaniu. Wpływa na pracę każdego członka i z niej wynikają wszystkie późniejsze decyzje organizacyjne. Z tego względu na nowo warto przemyśleć to jak aktualnie wygląda Nasza sekcja i jakie niesie to za sobą konsekwencje.\\\\
Sekcja pomimo swojego potencjału oraz uwagi zdobytej na początku roku nie odniosła sukcesu. Osoby wstępnie zapisanie do poszczególnych podsekcji nie wykazywały inicjatywy o co nie można ich winić, gdyż to po Naszej stronie stoi zapewnić odpowiednie warunki dla takich inicjatyw.\\Członkowie, którzy dołączyli do projektów trwających do teraz, pozostają w stosunkowej izolacji, zajmując się swoimi sprawami bez poszerzania swoich zagadnień. Zmniejsza to też szanse innych na niesienie pomocy.\\\\
Te właśnie problemy wynikają w znacznej mierze z budowy sekcji oraz braku inicjatywy samorządu na jej rozbudowanie. Możliwym sposobem ich rozwiązania jest wprowadzenie zupełnie nowego podziału sekcji oraz zmiana celów krótko- jak i długoterminowych.\\\\
\section{Wprowadzenie podziału na Organizację i Współpracę}
Mając doświadczenie z prowadzenia sekcji Organizacyjnej mogę z pewnością stwierdzić, że przy wiekszych grupach ludzi nie jesteśmy sobię w stanie poradzić bez zespołu pełniącego funkcję zarządzającą i kontrolującą. Taki też zespół potrzeba stworzyć, aby zapewnić stabilność.\\\\
Sam zespół nie jest jednak w stanie tworzyć sekcji bez jej członków. Poprzedni system zakładał, że każdy członek miał swoje konkretne zainteresowanie i przynależy do odpowiadającej temu podsekcji. Jak już wspomniałem, nie przyniosło to zadowalających efektów. Działo się tak ponieważ, podsekcji nie tylko izolowały od siebie ludzi, uniemozliwiając im współpracę z pozostałymi, ale także wiedzę. Ten kto przypisał się do odpowiedniej podsekcji w zamyśle zostawał w niej i miał ograniczoną pulę wiedzy i technologi do nauki.\\\\
Idealnym rozwiązaniem jest oddzielenie w sekcji części bezpośrednio samorządowej oraz uczniowskiej. Takie wydzielenie członków od ścisłego samorządu pozwoli przyciągnąć uwagę nie tylko nowych uczniów, ale także tych którzy zbyt stresowali się dołączeniem do ścisłego samorządu. Umożliwi nam to też posiadanie ludzi odpowiedzialnych za zasoby SU bez zrzucania tych zobowiązań na członków chcących realizować projekty w zupełnie innych technologiach.\\\\
\textbf{Dalej przedstawiam proponowany podział Obowiązków i Zadań.}

\section{Utworzenie małego sektora Organizacji projektów technicznych}
Proponowana struktura Sektora Organizacyjnego:\\\\
\textbf{1. Przewodniczący sekcji + 2. Zastępca Sekcji}\\
|\\
|\\
\textbf{3. Koordynator Projektów + 4. Koordynator Zasobów SU}\\
|\\
|\\
\textbf{5. (5-4) x Prowadzący projektów + 6. Prowadzący Githuba}\\

1. Przewodniczący sekcji:\\
Przewodniczący ma za zadanie kontrolować pracę Sektora Organizacyjnego oraz dobierać odpowiednich ludzi. Odpowiada za projekty z wykorzystaniem zasobów samorządowych oraz dla samorządu. Kontroluje pracę Koordynatorów, ich relację z prowadzącymi oraz status projektów. Dokonuje kontroli zasobów szkolnych i odpowiada za ich dostępność. Podejmuje decyzje odnośnie promocji, zmian struktury oraz członków sektora Organizacji.\\\\

2. Zastępca sekcji:\\
Zastępca sprawuje wszystkie zadania przewodniczącego, gdy tez nie jest w stanie ich realizować. Pomaga mu w zadaniach związanych z konrolą ludzi oraz zasobów. Odpowiada przed przewodniczącym ze swoich decyzji i działań. Jego zadaniem jest odciążenie przewodniczącego i sprawowanie roli jego prawej ręki podczas rozwiązywania problemów\\\\

3. Koordynator Projektów:\\
Koordynator Projektów ma za zadanie kontrolować, pomagać i analizować pracę Prowadzących projektów. Jego rolą jest zapewnienie stabilnośći sekcji, rozwiązywanie problemów w zarodku oraz tworzenie dobrej atmosfery pośród Prowadzących. Zbiera on informacje o postępach grup, wszelkie uwagi co do formy zebrań, prelekcji i samych projektów. Analizuje te informacje i przedstawia wnioski i/lub propozycje rozwiązań Przewodniczącemu w celu poprawy pracy całej sekcji.\\\\

4. Koordynator Zasobów SU:\\ 
Zasoby samorządu zawsze pozostawały tylko do wglądu Przewodniczącego. Sprawdzało się to dobrze, gdy głównymi jego zadaniami było właśnie nimi zarządzanie. Jednak jeśli chcemy rozbudować współprace w sekcji potrzebujemy osoby lub osób znających zastosowane technologie oraz umiejące z nich korzystać. Ma to na celu zmniejszenie stresu spoczywającego na przewodniczącym i prawdopodobieństwa popełnienia błędu. Wszystkie zasoby za które odpowiedziany będzie koordynator wymieniowe są na \url{https://app.tettra.co/teams/wisniowasu}.\\\\

5. Prowadzący projektów:\\
Jak przedstawione będzie późnie, z wględu na podział członków na projekty małe jak i duże potrzenujemy osób, organizaujących pracę tych ludzi. Aby zacząć przygotowywać naszych członków do pracy w zespołach i umiejętności dzielenia sie zadaniami potrzebujemy tworzyć tymczasowe wieksze grupy na czas trwania konkretnego projektu. Prowadzący właśnie ma za zadanie przedstawić założenia, przygotować plan działania, wyznaczyć zadania i poprzydzielać je do uczniów. Wyznacza on też ramy czasowe, metody komunikacji oraz raportowania o postępach.
Pełni kluczową rolę w sekcji, ponieważ od niego zależy jak przyjmą się nowe osoby i jak dobrze wypadny przy swoich pierwszych projektach. Powinien także pełnić rolę opiekuna, któremu można przedstawić swoje problemy i uwagi, a także zapytać o pomoc z zagadnieniami.\\\\

6. Prowadzący Githuba:\\
Jako że sekcja ta ma na celu tworzenie projektów informatycznych, podstawowym narzędziem pomagającym z kontrolą wersji oraz centralizacją zasobów jest Github. Gdy jednak okaże się, że mamy na tyle dużo projektów, niepotrzebnych repozytorów oraz nieuzupełnionych opisów, że trudno jest połapać się w pracy, potrzebować będziemy kogoś kto się tym zajmie. Prowadzący nasz zespół na Githubie powinien mieć odpowiednie zrozumienie jego działania i budowy. Ma za zadanie porządkowanie zespołu, ale także dodawanie nowych członków. Ta rola nie jest niezbędna, lecz w późniejszym rozwoju może okazać się potrzebna (prawdopodnie w innej postaci).\\\\\\
 
\textbf{Podział ten ma przede wszystkim na celu rozłożenie pracy na wiele osób. Pomoże to zmniejszyć stres i czas wymagany do operowania w sekcji. Jest to tylko propozycja i będę wdzięczny za wszelkie propozycje odnośnie zmiany nazwy, obowiązków lub całego stanowiska. Dalej przedstawiam formę funkcjonowania członków sekcji.}

\section{Utworzenie systemu Współpracy uczniów w projektach.}

Bazując na nauce wyciągniętej z zeszłego roku pracy potrzebujemy przedstawić bardziej obiecujący model sekcji dla uczniów. Głównym celem nowego systemu będzie umożliwienie uczniom dostosowanie projektów, grup lub par pod ich zainteresowania.\\\\
Aby każda technologia była traktowana na równi oraz każdy miał okazje nauczenia się tego co go interesuje, model projektów powinien skupiać się na umożliwieniu przedstawiania propozycji odnośnie całych projektów jak i ich elementów. Prowadzący projektów przedstawia cel w postaci produktu/usługi, a w interesie członków jest stworzenie planu jego wykonania. Ten proces, budowy od podstaw, zapewnia nam bezpieczeńswo, że projekty będą fascynowały i inspirowały członków od samego początku.\\\\
Najważniejszym w systemie współpracy będzie zrozumienie obu stron. Dlatego też ważnym jest dobór Prowadzących pod kątem charyzmy i komunikatywności. Musimy pamiętać, że bez członków sekcja nie funkcjonuje, a zatem są oni najważniejszą jej częścią.

\section{Podział Współpracy na różne wymiary grup.}


\chapter{Praca (Organizacja)}

\section{Wyznacznie struktury zarządzania projektami}
\section{Dobór osób zarzadzających pracą}
\section{Dostosowanie obowiązków i wymagań stanowisk do struktury}
\section{Wyznaczenie osób opiekujących się istniejącymi technologiami}

\chapter{Praca (Współpraca)}

\section{Wyznacznie metod współpracy uczniów}
\section{Organizacja spotkań sekcji}
\section{Prezentacja dostępnych projektów}
\section{Prezentacja możliwych warsztatów lub prelekcji}
\section{Oferty współpracy nad subprojektami/własnymi projektami}

\chapter{Dostęp i informacja}

\section{Wyznaczenie metod prezentowania zasobów sekcji}
\section{Stworzenie kampanii promującej udział w spotkaniach}
\section{Prezentacja dostępnych projektów, grafiku spotkań oraz warsztatów/prelekcji}
\section{Stworzenie formularza uczestnictwa w projektach lub nauczaniu}
\section{Umożliwienie proponowania projektów/wykorzystania technologii}

\chapter{Bonus}

\section{Zadania tygodniowe na dodatkowe oceny}
\section{Projekty wewnątrz-samorządowe}

\end{document}