\documentclass[9pt,a4paper]{report}

\usepackage[T1]{fontenc}
\usepackage[polish]{babel}
\usepackage[utf8]{inputenc}
\usepackage{graphicx}
\usepackage[export]{adjustbox}

\graphicspath{ {./media/} }

\usepackage[margin=1.2in]{geometry}

\usepackage{hyperref}
\hypersetup{
    colorlinks=true,
    linkcolor=black,
    filecolor=magenta,      
    urlcolor=blue,
}

\usepackage{amsfonts}
\usepackage{titlesec}

\titleformat{\chapter}[hang] 
{\normalfont\huge\bfseries}{\chaptertitlename\ \thechapter:}{1em}{} 
\titleformat*{\section}{\Large}
\begin{document}

\title{\Huge Projekt pracy i rozwoju sekcji technicznej samorządu uczniowskiego w latach 2020/2021.}
\author{Maciej Tracz \\\\Technikum Mechatroniczne nr 1 w Warszawie}
\date{Rok 2020}
\maketitle

\newpage
\tableofcontents
\newpage

\chapter{Struktura}

\section{Zmiana systemu}
Budowa sekcji stanowi najważniejszą rolę w jej funkcjonowaniu. Wpływa na pracę każdego członka i z niej wynikają wszystkie późniejsze decyzje organizacyjne. Z tego względu na nowo warto przemyśleć to jak aktualnie wygląda Nasza sekcja i jakie niesie to za sobą konsekwencje.\\\\
Sekcja pomimo swojego potencjału oraz uwagi zdobytej na początku roku nie odniosła sukcesu. Osoby wstępnie zapisanie do poszczególnych podsekcji nie wykazywały inicjatywy o co nie można ich winić, gdyż to po Naszej stronie stoi zapewnić odpowiednie warunki dla takich inicjatyw.\\Członkowie, którzy dołączyli do projektów trwających do teraz, pozostają w stosunkowej izolacji, zajmując się swoimi sprawami bez poszerzania swoich zagadnień. Zmniejsza to też szanse innych na niesienie pomocy.\\\\
Te właśnie problemy wynikają w znacznej mierze z budowy sekcji oraz braku inicjatywy samorządu na jej rozbudowanie. Możliwym sposobem ich rozwiązania jest wprowadzenie zupełnie nowego podziału sekcji oraz zmiana celów krótko- jak i długoterminowych.\\\\
\section{Wprowadzenie podziału na Organizację i Współpracę}
Mając doświadczenie z prowadzenia sekcji Organizacyjnej mogę z pewnością stwierdzić, że przy wiekszych grupach ludzi nie jesteśmy sobię w stanie poradzić bez zespołu pełniącego funkcję zarządzającą i kontrolującą. Taki też zespół potrzeba stworzyć, aby zapewnić stabilność.\\\\
Sam zespół nie jest jednak w stanie tworzyć sekcji bez jej członków. Poprzedni system zakładał, że każdy członek miał swoje konkretne zainteresowanie i przynależy do odpowiadającej temu podsekcji. Jak już wspomniałem, nie przyniosło to zadowalających efektów. Działo się tak ponieważ, podsekcji nie tylko izolowały od siebie ludzi, uniemozliwiając im współpracę z pozostałymi, ale także wiedzę. Ten kto przypisał się do odpowiedniej podsekcji w zamyśle zostawał w niej i miał ograniczoną pulę wiedzy i technologi do nauki.\\\\
Idealnym rozwiązaniem jest oddzielenie w sekcji części bezpośrednio samorządowej oraz uczniowskiej. Takie wydzielenie członków od ścisłego samorządu pozwoli przyciągnąć uwagę nie tylko nowych uczniów, ale także tych którzy zbyt stresowali się dołączeniem do ścisłego samorządu. Umożliwi nam to też posiadanie ludzi odpowiedzialnych za zasoby SU bez zrzucania tych zobowiązań na członków chcących realizować projekty w zupełnie innych technologiach.\\\\
\textbf{Dalej przedstawiam proponowany podział Obowiązków i Zadań.}

\section{Utworzenie małego sektora Organizacji projektów technicznych}
Proponowana struktura Sektora Organizacyjnego:\\\\
\textbf{1. Przewodniczący sekcji + 2. Zastępca Sekcji}\\
|\\
|\\
\textbf{3. Koordynator Projektów + 4. Koordynator Zasobów SU}\\
|\\
|\\
\textbf{5. (5-4) x Prowadzący projektów + 6. Prowadzący Githuba}\\

1. Przewodniczący sekcji:\\
Przewodniczący ma za zadanie kontrolować pracę Sektora Organizacyjnego oraz dobierać odpowiednich ludzi. Odpowiada za projekty z wykorzystaniem zasobów samorządowych oraz dla samorządu. Kontroluje pracę Koordynatorów, ich relację z prowadzącymi oraz status projektów. Dokonuje kontroli zasobów szkolnych i odpowiada za ich dostępność. Podejmuje decyzje odnośnie promocji, zmian struktury oraz członków sektora Organizacji.\\\\

2. Zastępca sekcji:\\
Zastępca sprawuje wszystkie zadania przewodniczącego, gdy tez nie jest w stanie ich realizować. Pomaga mu w zadaniach związanych z konrolą ludzi oraz zasobów. Odpowiada przed przewodniczącym ze swoich decyzji i działań. Jego zadaniem jest odciążenie przewodniczącego i sprawowanie roli jego prawej ręki podczas rozwiązywania problemów\\\\

3. Koordynator Projektów:\\
Koordynator Projektów ma za zadanie kontrolować, pomagać i analizować pracę Prowadzących projektów. Jego rolą jest zapewnienie stabilnośći sekcji, rozwiązywanie problemów w zarodku oraz tworzenie dobrej atmosfery pośród Prowadzących. Zbiera on informacje o postępach grup, wszelkie uwagi co do formy zebrań, prelekcji i samych projektów. Analizuje te informacje i przedstawia wnioski i/lub propozycje rozwiązań Przewodniczącemu w celu poprawy pracy całej sekcji.\\\\

4. Koordynator Zasobów SU:\\ 
Zasoby samorządu zawsze pozostawały tylko do wglądu Przewodniczącego. Sprawdzało się to dobrze, gdy głównymi jego zadaniami było właśnie nimi zarządzanie. Jednak jeśli chcemy rozbudować współprace w sekcji potrzebujemy osoby lub osób znających zastosowane technologie oraz umiejące z nich korzystać. Ma to na celu zmniejszenie stresu spoczywającego na przewodniczącym i prawdopodobieństwa popełnienia błędu. Wszystkie zasoby za które odpowiedziany będzie koordynator wymieniowe są na \url{https://app.tettra.co/teams/wisniowasu}.\\\\

5. Prowadzący projektów:\\
Jak przedstawione będzie późnie, z wględu na podział członków na projekty małe jak i duże potrzenujemy osób, organizaujących pracę tych ludzi. Aby zacząć przygotowywać naszych członków do pracy w zespołach i umiejętności dzielenia sie zadaniami potrzebujemy tworzyć tymczasowe wieksze grupy na czas trwania konkretnego projektu. Prowadzący właśnie ma za zadanie przedstawić założenia, przygotować plan działania, wyznaczyć zadania i poprzydzielać je do uczniów. Wyznacza on też ramy czasowe, metody komunikacji oraz raportowania o postępach.
Pełni kluczową rolę w sekcji, ponieważ od niego zależy jak przyjmą się nowe osoby i jak dobrze wypadny przy swoich pierwszych projektach. Powinien także pełnić rolę opiekuna, któremu można przedstawić swoje problemy i uwagi, a także zapytać o pomoc z zagadnieniami.\\\\

6. Prowadzący Githuba:\\
Jako że sekcja ta ma na celu tworzenie projektów informatycznych, podstawowym narzędziem pomagającym z kontrolą wersji oraz centralizacją zasobów jest Github. Gdy jednak okaże się, że mamy na tyle dużo projektów, niepotrzebnych repozytorów oraz nieuzupełnionych opisów, że trudno jest połapać się w pracy, potrzebować będziemy kogoś kto się tym zajmie. Prowadzący nasz zespół na Githubie powinien mieć odpowiednie zrozumienie jego działania i budowy. Ma za zadanie porządkowanie zespołu, ale także dodawanie nowych członków. Ta rola nie jest niezbędna, lecz w późniejszym rozwoju może okazać się potrzebna (prawdopodnie w innej postaci).\\\\\\
 
\textbf{Podział ten ma przede wszystkim na celu rozłożenie pracy na wiele osób. Pomoże to zmniejszyć stres i czas wymagany do operowania w sekcji. Jest to tylko propozycja i będę wdzięczny za wszelkie propozycje odnośnie zmiany nazwy, obowiązków lub całego stanowiska. Dalej przedstawiam formę funkcjonowania członków sekcji.}

\section{Utworzenie systemu Współpracy uczniów w projektach.}

Bazując na nauce wyciągniętej z zeszłego roku pracy potrzebujemy przedstawić bardziej obiecujący model sekcji dla uczniów. Głównym celem nowego systemu będzie umożliwienie uczniom dostosowanie projektów, grup lub par pod ich zainteresowania.\\\\
Aby każda technologia była traktowana na równi oraz każdy miał okazje nauczenia się tego co go interesuje, model projektów powinien skupiać się na umożliwieniu przedstawiania propozycji odnośnie całych projektów jak i ich elementów. Prowadzący projektów przedstawia cel w postaci produktu/usługi, a w interesie członków jest stworzenie planu jego wykonania. Ten proces, budowy od podstaw, zapewnia nam bezpieczeńswo, że projekty będą fascynowały i inspirowały członków od samego początku.\\\\
Najważniejszym w systemie współpracy będzie zrozumienie obu stron. Dlatego też ważnym jest dobór Prowadzących pod kątem charyzmy i komunikatywności. Musimy pamiętać, że bez członków sekcja nie funkcjonuje, a zatem są oni najważniejszą jej częścią.

\section{Podział Współpracy na różne wymiary grup.}

Jak pokazał nam zeszły rok szkolny, nie dając uczniom wyboru formy pracy, zmniejszamy ich zainteresowanie i nie dajemy możliwości wykazania się inicjatywami. Każdy członek powinien mieć swobodę w wyborze formy pracy, projektu oraz osób współuczestniczących. Aby jednak zapewnić takie możliwości musimy tworzyć projekty z zarówno dużymi i ambitnymi celami, jak i małymi i wprowadzającymi w świat informatyki. Proponowany przez mnie podział ma na celu jak najwiekszą organizację przy zachowaniu powyższych celów.\\\\

Formy podziału grup:
\begin{itemize}
\item Projekty zespołowe

Najważniejsza część nowej Współpracy z uczniami. Ich celem jest stworzenie apliakcji lub usługi w zespole ludzi posiadających rożne umiejętności na różnych poziamach zaawansowania. Mają one uczyć uczniów współpracy, terminowości oraz skrupulatności w wykonywaniu swoich zadań. Aby mogły być skutecznie prowadzene wymagają nadzoru osoby kontrolującej. Takie zadanie pełnią wcześniej wymienieni Prowadzący. Dzięki nim członkowie wiedzą jakie zadania do nich należą i jakie terminy ich obowiązują. Projekty te będą składać się z zespołów conajmniej 5 osób i conajmniej dwóch technologii\\
\item Projekty grupowe

Nie każdy projekt musi być duży i ambitny. Wszyscy zaczynaliśmy małymi krokami powoli zwiększając tempo nauki i rozwoju. Dlatego właśnie projekty średniej wielkości, wymagające 3-4 osób, wydzielamy od tych wiekszych. Mniejszym grupom prościej jest koordynować zadania i informować siebie o postepach. Dlatego też do tych projektów Prowadzący nie powinien znacznie ingerować. Wciąż należy do niego analiza postępów, rozmowy z członkami na temat współpracy oraz raportowanie, jednak nie powinni oni wyznaczać szczegółowych planów zadań ani odpowiednich im terminów. Ma to zapewnić swobodę pracy i nauczyć zarówno samodzielności jak i współpracy przy problemach.
\item Projekty osobiste

Jeśli jednak ktoś pragnie nauczyć się nowej technologii lub po prostu nabyć trochę cennej wiedzy, naszym zadaniem jest zapewnić do tego warunki. Udostępniając model projektów osobistych dla pojedyńczych osób lub par, dajemy im poczucie celowości i motywacji ograniczeniem czasowym. Prowadzący powinni służyć pomocną dłonią podczas tworzenia planu zadań oraz organizacji czasu. Jeśli uczeń wyraża zainteresowanie nabycia wiedzy od kogoś bardziej obeznanego, Prowadzący powinien przekazać taką informację do Koordynatora i rozważyć możliwe rozwiązania. W przypadku gdy dostępna jest osoba chcąca udostępnić swoją wiedze innym przydziela się ją do takiego projektu osobistego lub rozważa poprowadzenie prelekcji/warsztatów, gdy reszta członków wyrazi zainteresowanie. Projekty osobiste mogą być stosowane jako wstęp dla nowych członków, jako obeznanie z modelem pracy i organizacji w sekcji.
\end{itemize}

\chapter{Praca (Organizacja)}

\section{Dobór osób zarzadzających pracą}

Każdy plan, projekt czy pomysł potrzebuje osób wprowadzających go do życia. Nie mogą to być osoby pierwsze lepsze. Aby zapewnić rozwój i sukces sekcji musimy upewnić się, że osoby osadzone na konkretnych stanowiskach organizacyjnych będą w pełni zaangażowane i pełne inicjatywy do pracy. Musimy także uświadomić każdego potencjalnego kandydata o obowiązkach jak i korzyściach płynących z pełnienia takiej roli. Osoby odpowiednie do zarządzania powinny:\\
\begin{itemize}
\item Być charyzmatyczne
\item Lubić pracę z ludzmi
\item Umieć wyrazić swoje zdanie, ale też zrozumieć innych
\item Mieć pasję do organizacji pracy i/lub czasu
\end{itemize}
Jak można zauważyć w opisie poszczególnych pozycji w strukturze sekcji, nie wymagamy od tych osób szczególnej znajomości technologii jakie wykorzystywane są do tworzenia aplikacji bądz usług. Od nich zależy analiza pracy, zapewnianie dobrych relacji między członkami i tworzenie list zadań. Oczywiście znajomośc technologii i zagadnień znacznie pomaga w tworzeniu takich list, warto zaangażować samych członków do wyznaczania koniecznych zadań wraz z czasem potrzebnym na ich wykonanie. Do nas należy też zapewnić osobom zarządzającym miłej atmosfery i satysfakcji z osiągnięć.

\section{Dostosowanie obowiązków i wymagań stanowisk do struktury}

Przez wzgląd na to że jest to tylko zamysł/konspekt pracy i rozwoju sekcji może okazać się, że niektóre stanowiska będą bardziej intensywne od innych, będą wywierały zbyt dużo napięcia lub wręcz przeciwnie, dawały niedosyt ambitnym osobom. To też skłania mnie do elastycznego podchodzenia do obowiązków i zadań jakie przypisane są konkretnym osobom, o ile zachowują one integralność pracy i nie burzą hierarchi struktury Sektora Organizacji.\\\\

Ważną częścią rozwoju będzie zbieranie opinii uczniów na temat naszych postępów i ich doświadczeń z naszymi propozycjami. Na ich bazie będziemy wprowadzać poprawki na bazie kwartalnych spotkań poprawy warunków współpracy i semestralnych spotkań skierowanych na poprawę pracy całej sekcji. Ma to na celu pozwolenie każdemu na wyrażenie swojego zdania o prosperacji sekcji. Jeszcze raz warto przypomnieć, że sekcja ta w przeciwieństwie do pozostałych składa się z dwóch zupełnie różnych sektorów i nawet jeśli nasi koledzy kierujący pracą czują się usatysfakcjonowani, to bez pozostałych uczniów nie pójdziemy daleko. Każdy musi czuć się zauważany i zadowolony z swojej pracy, aby był zmotywowany do dalszego poszerzania horyzontów.\\

\section{Wyznaczenie osoby/osób opiekujących się istniejącymi technologiami}

Mimo niewielkiego rozmiaru naszej organizacji, jaką jest Samorząd Uczniowski, posiadamy znaczne środowisko sieciowe oraz technologiczne udostępniające usługi pokroju strony głównej SU jak i wewnetrznych zasobów szkoleniowych. Jako, że wykorzystujemy ponad 20 technologii, z których każda wymaga odrobiny doświadczenia, potrzebujemy zdjąć z barków przewodnicącego plakietkę informatyka od wszystkiego. Sam nie jestem narazie w stanie obsłużyć infrastruktury pozostawionej mi przez Piotrka, nawet jeśli przez ten rok zadeklarował się do administrowania nią. Jednak nie tyle moje umiejętności są tu obawą lecz przyszłe lata i kolejni przewodniczący. Może okazać się tak, że po moim odejściu z tej szkoły nie znajdziemy kogoś nowego na pstryknięcie palcami. Dlatego też, za wczasu, musimy stworzyć zarówno rolę osoby odpowiedzialnej za infrastrukturę jak i program uczący nowych kandydatów jak nią zarządzać.\\\\
Oboje z Piotrkiem uważamy tą odpowiedzialność za kluczową, jako że wiekszość naszych akcji  skupia się na wykorzystaniu technologii informatycznych mających korzenie w naszej szkole. Jednak z względu na zaawansowanie, ilość i bezpieczeństwo dostępu do zasobów samorządowych, dobrym było by podzielenie tych zadań na paru ludzi. Niesie to za sobą wiele korzyści takich jak:\\
\begin{enumerate}
\item Zmniejsza stres na jaki wystawione są osoby odpowiedzialne
\item Zwiekszą bezpieczeństwo przed pomyłką
\item Wciąga więcej osób w świat zastosowanych technologii
\item Daje nam pole do manewrów w przypadku choroby lub nieobecności jednej z osób odpowiedzialnych.
\end{enumerate}

Temat ten będę chciał poruszyć na najbliższym zebraniu zarządu jako iż jest ważną częścią pracy samego samorządu. Oczywiście jestem świadom niebezpieczeństw jakie niesie za sobą posiadanie większej liczby osób mających dostęp do danych. W tym właśnie aspekcie pracujemy z Piotrkiem nad rozwiązaniami pozwalającymi na dostęp przez MFA lub Token Authorisation. W przypadku wszelkich propozycji, uwag lub zastrzeżeń proszę o kontakt na facebooku lub zachowanie ich na zebranie.

\chapter{Praca (Współpraca)}

\section{Wyznacznie metod współpracy uczniów}
\section{Organizacja spotkań sekcji}
\section{Prezentacja dostępnych projektów}
\section{Prezentacja możliwych warsztatów lub prelekcji}
\section{Oferty współpracy nad subprojektami/własnymi projektami}

\chapter{Dostęp i informacja}

\section{Wyznaczenie metod prezentowania zasobów sekcji}
\section{Stworzenie kampanii promującej udział w spotkaniach}
\section{Prezentacja dostępnych projektów, grafiku spotkań oraz warsztatów/prelekcji}
\section{Stworzenie formularza uczestnictwa w projektach lub nauczaniu}
\section{Umożliwienie proponowania projektów/wykorzystania technologii}

\chapter{Bonus}

\section{Zadania tygodniowe na dodatkowe oceny}
\section{Projekty wewnątrz-samorządowe}

\end{document}